\section{Motivation}
The amount of data generated by web applications are growing larger and larger.
Making all the data avaiable for users requires specialized search engines.
Most of todays search engines implements a technique called TF/IDF when searching for documents.
This technique is effective and returns relevant search results most of the time,
but with the ever growing data size web applications tries to return more relevant search results.




Search engines today strives to deliver fast and relevant search results.
Most users may quickly notice whether the search results are relevant or not.
A second important factor for interactive tasks is speed.
Interactive services have a requirement to deliver results within 100ms \cite{google-latency}.

Today an increasing number of software companies try to personalize their services.
A few examples of personalized services are Facebook's News Feed\footnote{\url{https://newsroom.fb.com/news/category/news-feed-fyi/}},
Netfix's movie suggestion \cite{netflix-recommendation} and Spotify's Discover Weekly\footnote{\url{https://community.spotify.com/t5/Spotify-Community-Blog/On-Track-Discover-Weekly/ba-p/1456790}}.

High latency may lead to users abandoning your site and revenue is easily lost.
Returning results fast is more important compared to the number of results returned.
While conducting latency experiments, Google found that an increased latency of 0.5 seconds, leads to 20\% less traffic \cite{google-marissa}.

This master thesis builds on and extends the work done by Juul Arthur R Rudihagen \cite{master-thesis}.
Rudihagen researched how an instant personalized search could be achieved.
The research achieved a personalized search which returned results of higher relevance to the user.
However, the results from the work showed that the implemented method did not meet the requirement of interactivity \cite{master-thesis}.

Furthermore, this report examines how a fast and scalable search may be achieved, using the research done by Rudihagen as a baseline.
His search implementations had limits both in latency and scalability.
A major part of the latency in Rudihagen's report originated from multiple round trips from the web server to the database and the search engine.

[??]This should be somewhere in motivation.
Google\footnote{\url{https://www.google.com}} and Bing\footnote{\url{https://www.bing.com/}}
were the two largest search engines in 2016\footnote{\url{https://www.netmarketshare.com/search-engine-market-share.aspx?qprid=4&qpcustomd=0&qpstick=0&qpsp=2016&qpnp=1&qptimeframe=Y}},
according to NetMarketShare.
However, neither Google and Bing are open source.

In this report we will look at how the latency may be reduced and scalability increased,
and based on Rudihagen's work this report have derived the research questions in the next section.
