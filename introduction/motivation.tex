\section{Motivation}
The amount of data generated by web applications are continuously increasing.
Making all the data avaiable for users, requires specialized search engines.
Most of today's search engines implement a technique called term frequency-inverse document frequency when searching for documents.
This technique is effective, and returns relevant search results most of the time,
but with the ever growing data size web applications strive to deliver even more relevant search results.
Personalization is a common approach to deliver more relevant search results.
If a user in Trondheim search for the word \texttt{resturant},
resturants in Trondheim are most likely more relevant compared to resturants in Oslo.
Today an increasing number of software companies try to personalize their services.
A few examples of personalized services are Facebook's News Feed \cite{facebook},
Netfix's movie suggestion \cite{netflix-recommendation} and Spotify's Discover Weekly \cite{spotify}.

A second important factor for interactive tasks such as search, are speed.
Interactive tasks have a requirement of 100 ms before the user recognizes the delay \cite{google-latency}.
This means that search results have to be available within 100 ms from the user typed the query.
High latency may lead to users abandoning your site, and revenue is easily lost.
Returning results fast is more important compared to the number of results returned.
While conducting latency experiments, Google found that users said that they wanted 30 search results instead of 10 \cite{google-marissa}.
However,
their tests also showed that users who recieved 30 search results gave less traffic compared to users who recieved 10 search results.
The difference between the search results were load times.
10 search results took 0.4 seconds to load and 30 search results took 0.9 seconds to load.
It shows that speed is an important factor in search.

Personalized search engines already exists.
Google\footnote{\url{https://www.google.com}} and Bing\footnote{\url{https://www.bing.com/}}
were the two largest search engines in 2016 according to NetMarketShare \cite{search-engine-rank}.
However, neither Google nor Bing are open source.

A common approach to improve the search result is to extend the user's query with more terms,
or adding additional information like the user's position.
Research has found it diffiucult to select good terms to expand \cite{pseudo-relevance-invalid}.
On the other hand, tags have often been found to provide good expansion terms \cite{ir-hashtag}.
In this master thesis query expansion will be implemented together with a technique called pseudo-relevance.
An important requirement for the implementation will be to deliver search results within 100 ms and that the implementation is scalable.

This master thesis builds on and extends the work done by Rudihagen \cite{master-thesis} and my project report \cite{project-report}.
Rudihagen researched how an instant personalized search could be achieved.
The research achieved a personalized search which returned results of higher relevance to the user compared to TF-IDF.
However, the results from the work showed that the implemented method did not meet the requirement of interactivity \cite{master-thesis}.
An important factor for the latency was the number of round trips between the web server and the search engine.
In my project report \cite{project-report}, an implementation is described which reduced the number of round trips between the web server and the search engine from four to two.
However, the implementation may be improved even further by implementing query expansion with pseudo-relevance directly on the search engine.

This master thesis will explore how the implementation described by Rudihagen and in my project report can be improved even further.
The next section describes three research questions based on the motivation introduced in this section.
