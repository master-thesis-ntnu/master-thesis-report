%\section{Research Question Evaluation}
%\label{sec:reasearch-questions-evaluation}
%\begin{enumerate}
  %\item How to provide instant personalized recommendations on a cold start based on the query typed?
%  \item \textbf{How to achieve more relevant search results based on a query from a user?} \newline
%  As mentioned in section \ref{sec:problem-specification} the main focus areas were scaling and latency.
%  Therefore, the relevance was never measured.
%  Based on the work by Rudihagen,
%  this thesis assumes that query expansion returns more relevant results compared to the baseline search.

%  \item\label{rq:scaling} \textbf{How to make the search recommandation scale with an increasing amount of data?} \newline
%  As mentioned earlier, Elasticsearch is proven to scale to petabytes of data \cite{elasticsearch-scale},
%  if configured correctly, and was thus used as the search engine for this project report.
%  Rudihagen's implementation also used Elasticsearch, and was configured with one index for each user.
%  This configuration is fine for most cases, but according to the documentation this does not scale well if you have a large user base \cite{elasticsearch-indices}.

%  Elasticsearch require proper tuning and configuration.
%  Without testing the solution on a large data we cannot conclude that the implementation would scale,
%  but the current implementation have potential to scale.

%  \item\label{rq:latency} \textbf{How to implement techniques to deliver more relevant search results and at the same time fulfill the interactive requirements?} \newline
%  The results described in section \ref{sec:results} shows that the query expansion plugin for Elasticsearch is well within the limit for interactive applications.
%  With a search result size of 10 the
%  From the results we can see that the implementation are delivering results in about 16 ms, in a test environment.
%  However, the query expansion implementation has a latency which is about 2 times greater than the baseline.
%  For real world use cases, this might exceed the latency to above the interactivity limit.
%\end{enumerate}

% \section{Notes}
% - Show example results. Same query, but one with query expansion and one without.
