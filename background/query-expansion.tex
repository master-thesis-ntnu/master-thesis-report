\section{Query Expansion}
\label{sec:query-expansion}

When a user searches using the query "Super Bowl" the day after the sport event has taken place,
it is likely that the user wants information about the event from the day before.
The query "Super Bowl" is likely to also return documents from previous years.
If the search engine could be able to notice that recent documents also contains the term "2016,"
the extra term could be added to the query.
The new query "Super Bowl 2016" is likely to rank documents from this year's superbowl higher, as a result of the extra term.
This technique is called query expansion.

The idea behind query expansion is to add more terms to the user's query, and then use the extended query on the search engine.
According to literature a query expanded search does improve the results \cite[ch. 5]{ir-book}.
Even though research shows promising results, query expansions require explicit information which in practice often is difficult to acquire.
On the other hand, according to Efron \cite{ir-hashtag} hashtags provides an excellent way to acquire the explicit information needed for query expansion.

There are different methods of query expansion, and this report describes one technique called Kullback-Leibler divergence.
The implementation is described in chapter \ref{ch:approach}.

\subsection{Kullback-Leibler Divergence}
\textit{Kullback-Leibler divergence} (KL) measures how well distribution P(t) represents the distribution Q(t).
The variables in distribution P(t) and Q(t) is explained in the bullet points bellow.

\begin{itemize}
	\item \textit{numberOfTimesInTopKDocuments} is the number of times a term is present in the top-k documents
	\item \textit{numberOfTermsInTopKDocuments} is the number of terms in total in the top-k documents
	\item \textit{totalNumberOfTimesInCollection} is the total number of times a term is present in the data collection
	\item \textit{totalNumberOfTermsInCollection} is the total number of terms in the data collection
\end{itemize}

Equation \ref{eq:kl-distribution-p} explains how to calculate the distribution P(t),
and equation \ref{eq:kl-distribution-q} explains how to calculate distribution Q(t).

% Kullback-Leibler P distribution
\begin{cequation}[H]
	\begin{equation}
		\mathbf{P} = \frac{numberOfTimesInTopKDocuments}{numberOfTermsInTopKDocuments}
	\end{equation}
	\caption{}
  \label{eq:kl-distribution-p}
\end{cequation}

% Kullback-Leibler Q distribution
\begin{cequation}[H]
	\begin{equation}
		\mathbf{Q} = \frac{totalNumberOfTimesInCollection}{totalNumberOfTermsInCollection}
	\end{equation}
	\caption{}
  \label{eq:kl-distribution-q}
\end{cequation}

Computing the Kullback-Leibler divergence for a term is given by equation \ref{eq:kl-divergence}.

% Kullback-Leibler Distance Equation
\begin{cequation}[H]
	\begin{equation}
		\mathbf{KL}_D[P(t), Q(t)] = P(t)*\log{[\frac{P(t)}{Q(t)}]}
	\end{equation}
	\caption{Kullback-Leibler Divergence.}
  \label{eq:kl-divergence}
\end{cequation}

\subsubsection{Kullback-Leibler Divergence Example}
To illustrate how KL score is calculated, an example for the search term "sky" is displayed.
The expanded query may consist of up to 5 terms.
To keep this example short, only the top 5 terms is calculated and the term "sky" is excluded from the calculations.

Table \ref{tbl:kl-counts} contains information extracted using the data set described in section \ref{sec:dataset}.
Using equation \ref{eq:kl-divergence} and the information in table \ref{tbl:kl-counts}, the KL score for each term are calculated and is shown in table \ref{tbl:kl-score}.
The table \ref{tbl:kl-score} lists the terms descending according to their score.
From the original query we have the term "sky" and we may use 4 additional terms to complete the expansion.
The top 4 terms are "clouds", "2016", "blue" and "water",
which results in an expanded query containg the terms: "sky", "clouds", "2016", "blue" and "water".


\begin{table}
	\centering
    \begin{tabular}{|l|p{30mm}|p{30mm}|p{30mm}|p{30mm}|}
		\hline
    \textbf{Term}   & \textbf{numberOf- \newline TimesIn- \newline TopK- \newline Documents} &
		\textbf{numberOf- \newline TermsIn- \newline TopK- \newline Documents} &
		\textbf{totalNumber- \newline OfTimesIn- \newline Collection} &
		\textbf{totalNumber- \newline OfTermsIn- \newline Collection} \\ \hline
    blue   & 1                            & 42                           & 14                             & 298,962                         \\ \hline
    2016   & 2                            & 42                           & 143                            & 298,962                         \\ \hline
    clouds & 3                            & 42                           & 31                             & 298,962                         \\ \hline
    sea    & 1                            & 42                           & 34                             & 298,962                         \\ \hline
    water  & 1                            & 42                           & 24                             & 298,962                         \\ \hline
\end{tabular}
	\caption{The returned numbers for each of the top 5 terms excluding the term "sky".}
	\label{tbl:kl-counts}
\end{table}

\begin{table}
	\centering
    \begin{tabular}{|l|l|}
		\hline
    \textbf{Term}   & \textbf{Score}   \\ \hline
    clouds & 0.46678 \\ \hline
    2016   & 0.21908 \\ \hline
    blue   & 0.14836 \\ \hline
    water  & 0.13553 \\ \hline
    sea    & 0.12723 \\ \hline
    \end{tabular}
	\caption{KL divergence score of each term.}
	\label{tbl:kl-score}
\end{table}
