\section{Relevance Feedback}
The idea behind \textit{relevance feedback} is to use the result from the initial query to extract relevant information from the top-k documents.
Once the information is extracted, a new query is executed with extracted information.
Results from the second query are returned to the user.
The assumption is that the second query returns documents which are more relevant to the user.

In \cite{ir-book} the authors define \textit{relevance feedback} as: "when the user explicitly provides information on relevant documents to a query,"
and \textit{query expansion} as: "when information related to the query is used to expand it" \cite[p. 177]{ir-book}.
In other words to use \textit{relevance feedback}, input from the user is needed.
For example the user could be given the task to mark whether the documents are relevant or not.
In practice it is often difficult for a user to determine the result's relevance.
For \textit{query expansion} information like position and tags may be used to expand the query.
A more detailed explanation of query expansion is described in section \ref{sec:query-expansion}.

\textit{Relevance feedback} is divided into three main categories \textit{explicit feedback}, \textit{implicit feedback} and \textit{pseude-relevance feedback},
and is introduced in the next two subsections.

\subsection{Explicit vs Implicit Feedback}
\textit{Explicit feedback} data are retrieved directly from user interaction.
An example would be if the user selects the section "graphic cards" in an online store.
From the interaction the user explicitly states that the search should only contain graphic cards.
Another approach is to use data from a user search.
If a user clicks on a search result, the result may be regarded as relevant.
Even though the result may not be relevant, it is a good indication.
The problem with explicit feedback, is that it requires interaction from the user.

\textit{Implicit feedback} on the other hand, does not require any involvement from the user.
Examples of implicit user data are collecting the documents from a search result that  are opened by a user,
and measure time spent viewing a document.

\subsection{Pseudo-Relevance Feedback}
Retrieval of data to use relevance feedback requires either explicit or implicit user interaction.
Manually involving the user in the search is undesireable.
To avoid this, an approach called pseudo-relevance feedback can be used.
Using implicit feedback requires a system which does the data collection and post process the information.
Pseudo-relevance, on the other hand, uses information from the first search, and thus leads to a simpler implementation.

Often the top-k documents are used to find pseudo-relevance for query expansion.
However, the top-k documents are in many cases not relevant, and thus not suitable data for query expansion \cite{pseudo-relevance-invalid}.
Section \ref{sec:query-expansion} describes a method to extract information from the top-k documents which is regarded as relevant information.
