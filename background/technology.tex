\section{Technology}
The experiment in this project report utilized a web server and a search engine as the backend.

NodeJS\footnote{\url{https://nodejs.org}} v7 was chosen as the webserver.
NodeJS was chosen because the author has knowledge of the technology,
and it contains a rich package manager called NPM.
By utilizing open source libraries through NPM, more time could be spent implementing the algoritms for query expansion.
Inside NodeJS lies the V8\footnote{\url{https://developers.google.com/v8/}} JavaScript engine.

Elasticsearch\footnote{\url{https://www.elastic.co/products/elasticsearch}} v5 were utilized as the search engine.
At the time of writing Elasticsearch is a popular open source search engines, which has proven the ability to scale up to petabytes of data \cite{elasticsearch-scale}.
Elasticsearch is open source and built on top of Lucene\footnote{\url{https://lucene.apache.org/}}.
Lucene is the search engine itself,
and Elasticsearch provides functionality for distribution and a REST API interface.

\subsection{Lucene}

\subsection{Elasticsearch}
