\chapter{Evaluation}
\label{ch:evaluation}

\section{Experimental Setup}

\subsection{Lucene Experiment}
The Lucene experiments starts to build the index with all the photo data.
To make the experiment simpler only the fields: tags, title and url are stored.

\subsection{Elasticsearch Experiment}
The Elasticsearch experimental setup contains two main components, a web server and a search engine.
The web server is implemented in NodeJS and the search engine used is Elasticsearch.

As large applications ofter use cloud providers, the tests also needed to be conducted using cloud providers.
The requirement set for the cloud provider was: need to be videly used, have servers in Europe and provide VPS services.
Possible providers were: Amozon Web Services\footnote{\url{https://aws.amazon.com/}},
Google Cloud Platform\footnote{\url{https://cloud.google.com/}} and Digital Ocean\footnote{\url{https://www.digitalocean.com/}}.
Google Cloud Platform were chosen as the service provider, as you have more flexibility to choose between number of cores and the memory size,
the author had both knowledge to the platform and they gave away free credits[?? Maybe give a better reason?].

Tests were conducted using two Google Compute Engine instances.
The instance running Elasticsearch had the following specifications: 2 vCPUs, 10 GB memory and 20 GB SSD.
Elasticsearch's documentation\footnote{\url{https://www.elastic.co/guide/en/elasticsearch/guide/current/hardware.html}} suggest that memory will be the most important resource in most use cases.
As a result more memory were chosen over the number of CPUs.
An important setting in Elasticsearch is the heap memory usage.
By default the heap memory size is set to 1 GB.
The Elasticsearch heap size were set to 5 GB in the test environment.
According to the Elasticsearch documentation
Set the heap size to half of the memory\footnote{\url{https://www.elastic.co/guide/en/elasticsearch/guide/current/heap-sizing.html  }}.

Node JS
4vCPUs, 4GB memory
10 GB SSD

location europe-west1-c

\subsection{Data Set}
\label{sec:dataset}

\subsection{Performance Metrics}

\section{Results}
\label{sec:results}
The results from the experiments in the project report [??],
showed that the query expansion implementation had about 2 times longer latency compared to the baseline implementation.
The latency were measured from the request left the user to the response from the server arrived.

All the results show that the first request is often the slowest.
After the initial request the respons is cached by Lucene and makes all the subsequent requests a lot faster.

\subsection{Lucene Results}
About 50\% increased latency with the Lucene implementation.
Can see that Lucene heavily caches search result.
The first initial results often is a lot slower than the subsequent searces.

\section{Discussion}

\section{Research Question Evaluation}
