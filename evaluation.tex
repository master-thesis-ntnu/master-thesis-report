\chapter{Evaluation}
\label{ch:evaluation}

\section{Experimental Setup}
- Why are there two experiments.
- Author discovered Elasticsearch's plugin API after implementing query expansion in lucene after looking into how to move the implementation into Elasticsearch.
- Elasticsearch plugin API have very limited documentation.
- Hard to implement.

\subsection{Lucene Experiment}
The Lucene experiments starts to build the index with all the photo data.
To make the experiment simpler only the fields: tags, title and url are stored.

\subsection{Elasticsearch Experiment}
The Elasticsearch experimental setup contains two main components, a web server and a search engine.
The web server is implemented in NodeJS and the search engine used is Elasticsearch.

As large applications ofter use cloud providers, the tests also needed to be conducted using cloud providers.
The requirement set for the cloud provider was: need to be videly used, have servers in Europe and provide VPS services.
Possible providers were: Amozon Web Services\footnote{\url{https://aws.amazon.com/}},
Google Cloud Platform\footnote{\url{https://cloud.google.com/}} and Digital Ocean\footnote{\url{https://www.digitalocean.com/}}.
Google Cloud Platform were chosen as the service provider, as you have more flexibility to choose between number of cores and the memory size,
the author had both knowledge to the platform and they gave away free credits[?? Maybe give a better reason?].
Tests were conducted using two Google Compute Engine instances.
Web services always strives to place the servers as close to the users as possible.
To make the experiment simulate a real world scenario both the instances were placed in the reagion called \textit{europe-west1-c}.

\subsubsection{Elasticsearch Instance}
The instance running Elasticsearch had the following specifications: 2 vCPUs, 10 GB memory and 20 GB SSD.
Elasticsearch's documentation\footnote{\url{https://www.elastic.co/guide/en/elasticsearch/guide/current/hardware.html}}
suggest that memory will be the most important resource in most use cases.
As a result more memory were chosen over the number of CPUs.
An important setting in Elasticsearch is the heap size.
By default the heap memory size is set to 1 GB, but were changed to 5 GB in the test environment.
A logical asumption would be to set the Elasticsearch to use all the available memory, except the memory needed for the operating system.
However, Elasticsearch's underlying structure Lucene also needs memory.
Lucene stores the data in separate files.
The datastructure inside the files are immutable, which makes them optimized for caching.
With this strategy Lucene optimizes the underlying operating system's eager to hold small and often used files in memory.
According to the Elasticsearch documentation\footnote{\url{https://www.elastic.co/guide/en/elasticsearch/guide/current/heap-sizing.html}}
the heap size should be set to 50\% or less, of the available memory.

Most operating systems today also comes with swapping turned on by default.
If the operating system decides to swap it would significantly reduce the performance.
To avoid the problem swapping were turned off on the Elasticsearch instance.

\subsubsection{NodeJS Instance}
The instance running NodeJS had the following specifications: 4 vCPUs, 4 GB memory and 10 GB SSD.
On the web server we want to be able to handle as many requests as possible.
The number of requests the server is able to handle are closely linked to the number of cores.
That is why the Node.js instance has more cores at the cost of less memory.

Node.js is by design single threaded, which would make 3 of the cores on the Node.js server being idle.
However, this problem can be solved by using tool called pm2\footnote{\url{http://pm2.keymetrics.io/}}.
pm2 has a feature called \textit{cluster mode}, which may spawn multple Node.js instances.
To allow maximum CPU utilization, pm2 can be configured to spawn as many Node.js instances as the number of cores.

\subsection{Data Set}
\label{sec:dataset}
The data set consits of photo data gathered from the Flickr API \footnote{\url{https://www.flickr.com/services/api/}}.

All the data are gathered over the periode january to may.
The data set used in the experiment consists of [??] photos and [??] tags.

Analysing the data set discovered that each photo have an average of [??] tags.
A total of [??] images have no tags and [??] have atleast 1 tag.

[?? move to approach?]
Elasticsearch has two different mapping options, dynamic mapping and static mapping.


\subsection{Performance Metrics}

\section{Results}
\label{sec:results}
The results from the experiments in the project report [??],
showed that the query expansion implementation had about 2 times longer latency compared to the baseline implementation.
The latency were measured from the request left the user to the response from the server arrived.

All the results show that the first request is often the slowest.
After the initial request the respons is cached by Lucene and makes all the subsequent requests a lot faster.

\subsection{Lucene Results}
About 50\% increased latency with the Lucene implementation.
Can see that Lucene heavily caches search result.
The first initial results often is a lot slower than the subsequent searces.

\subsection{Elasticsearch Experiment Results}
To evaluate the performance of the plugin developed a few tests were conducted.
All of the tests were done in two different ways,
one prewaring the cache and one without prewaring the cache.

\section{Discussion}

\section{Research Question Evaluation}

\section{Notes}
During testing the same query sometimes gave different results.
Because of this: https://www.elastic.co/blog/understanding-query-then-fetch-vs-dfs-query-then-fetch
