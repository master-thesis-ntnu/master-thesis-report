\subsection{Algorithm}
The algorithm used for query expansion is equal on both platforms,
but they have some platfroms specific differences which are explained in their respective subsections.

The algorithm starts by sending a term search to the search engine.
The result from the initial search are often reffered to as top-k documents,
where k stands from the number of documents in the results.
The photos from the result are then looped through to extract all the tags.
Each tag is stored in an hash map for fast retrieval.
On every iteration the tag are checked against the hash map.
If the term does not exists,
the term is added to the hash map.
The key is the hash map itself,
and the value is an object which stores information about the total number of times the term occurs in the whole collection,
the total number of terms in total in the collection and the number of times the term is present in the top-k documents.
In the opposite case where the tag already exists, the term counter for the current term is incremented.
the counter for the number of times the term appears in the top-k documents are incremented by one.

On each iteration a search engine lookup is necessary to retrieve the number of times a certain term appears in the whole collection,
but as the term information is stored it is only necessary when a new term appears.


The initial thought where to calculate the KL-score for each term and skip the step

The information about how many times a tag appears in the collection are retrieved from the search engine.
The number of search engine accesses depends on the number of unique tags in the top-k documents.

\begin{algorithm}
  \begin{algorithmic}
    \Require{Search terms from the user is defined as $searchTerms$}
    \State $initialSearchResult \gets \Call{termsSearch}{searchTerms}$
    \State $termsData \gets [ ]$
    \State $sortedTerms \gets [ ]$

    \For {$photos$ in $initialSearchResult$}
      \For {$term$ in photos}
        \State $numberOfTimesInCollection \gets \Call{getNumberOfTimesInCollection}{term}$
        \State $numberOfTimesInCollection \gets \Call{getNumberOfTimesInCollection}{term}$
        \State \Call{insertTermsData}{term, numberOfTimesInCollection, numberOfTermsInCollection}
      \EndFor
    \EndFor
    \For {$term$ in $termsData$}
      \State $\Call{calulate}$
      \Call{insertTerm}{sortedTerms, term}
    \EndFor

  \end{algorithmic}
  \caption{Algorithm used in the Lucene implementation.}
\end{algorithm}
