\subsection{Lucene Architecture}
This subsection describes how query expansion were implemented using the Apache Lucene java library.
The Lucene implementation were done using version 6.4.0 of Lucene \cite{lucene-documentation}.
To avoid Lucene using the disk the index were kept in memory,
which implemented by utilizing a class in Lucene called \texttt{RAMDirectory}.

There may be possible to achieve better performance by implementing Java optimizations,
but that is outside the scope of this master thesis and were thus never investigated.

\subsubsection{Indexing}
Lucene have multiple data types which may be stored,
but the types used in this implementation were \texttt{StringField}, \texttt{LongPoint} and \texttt{TextField}.
All the photo tags were indexed using the \texttt{TextField}.
The two fields \texttt{TextField} and \texttt{StringField} are quite similar,
but have some important differences.
Listing \ref{lst:lucene-string-field} displays the \texttt{StringField} configuration code from Lucene's Github repository\footnote{\url{https://github.com/apache/lucene-solr/blob/master/lucene/core/src/java/org/apache/lucene/document/StringField.java}}.
The first function \texttt{setIndexOptions(IndexOptions.DOCS)},
configures the inverted index to only store which documents contains the term.
Term frequencies and vector are not stored with this option.
\texttt{setStored(true)} tells the Lucene index to store the original value.
Lastly, the function \texttt{setTokenized(false)} informs Lucene to store the string value as a single token.

\begin{lstlisting}[language=java, caption={Lucene's \texttt{StringField} index configuration.}, label={lst:lucene-string-field}]
  setStored(true);
  setTokenized(false);
  setIndexOptions(IndexOptions.DOCS);
\end{lstlisting}

Listing \ref{lst:lucene-text-field} shows the \texttt{TextField} configuration code from Lucene's Github repository\footnote{\url{https://github.com/apache/lucene-solr/blob/master/lucene/core/src/java/org/apache/lucene/document/StringField.java}}.
The difference between the \texttt{StringField} and the \texttt{TextField} code are line two and three.
\texttt{setTokenized(true)} tells the Lucene analyzer emit a token for each word in the given string.
Most important is line three which tells Lucene to store data about which documents contains a given term,
a given terms frequncy in each document and the terms position in the original text.
Calculating the KL-score of a term requires the terms total frequency across all documents.

\begin{lstlisting}[language=java, caption={Lucene's \texttt{TextField} index configuration.}, label={lst:lucene-text-field}]
    setStored(true);
    setTokenized(true);
    setIndexOptions(IndexOptions.DOCS_AND_FREQS_AND_POSITIONS);
\end{lstlisting}

\subsubsection{Query Expanded Search}
Figure shows a sequence diagram for the Lucene implementation.
As the sequence diagram shows, the java program first recieves the query.
The query is then sent to the Lucene index as a multi term query.
Lucene then returns the top-k documents from the index.
Calculating the KL-scores requires information about how many times each term appears in the collection,
and the total number of terms in the collection.
As Lucene exposes a low level interface,
this information may be retrieved directly from the inverted index.
The function \texttt{totalTermFreq} from the class \texttt{IndexReader} returns the number of times a term is present across all documents.
The inverted index also holds information about the total number of terms in the collection.
This number is retrieved by using the function \texttt{getSumTotalTermFreq} from the same \texttt{IndexReader} class.
After all the necessary information have been gathered the KL-score is calculated,
and the new expanded search terms are generated.
Finally, the multi term search are sent to the index.
The final result are then returned back to the client.

\begin{figure}[h!]
  \centering \includegraphics[width=0.9\linewidth]{img/sequence-diagram-lucene.png}
  \caption{Sequence diagram for the Lucene implementation.}
  \label{fig:sequence-diagram-lucene}
\end{figure}
