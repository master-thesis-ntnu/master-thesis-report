\chapter{Background}
\label{ch:background}

\section{Open Source Search Engine Solutions}
Google\footnote{\url{https://www.google.com}} and Bing\footnote{\url{https://www.bing.com/}}
were the two largest search engines in 2016\footnote{\url{https://www.netmarketshare.com/search-engine-market-share.aspx?qprid=4&qpcustomd=0&qpstick=0&qpsp=2016&qpnp=1&qptimeframe=Y}}.
However, neither Google and Bing are open source.
Making large amounts of data available for users to search requires search engines.
At the time of writing there exists open source alternatives like Elasticsearch\footnote{\url{https://www.elastic.co/products/elasticsearch}},
Solr\footnote{\url{http://lucene.apache.org/solr/}} and Xapian\footnote{\url{https://xapian.org/}}.

Some of the search engines contains query expansion implementations,
but they are limited to synonyms expansion.
None of the open source databases deliver personalized search capabilities.

\section{Technology}

\subsection{Lucene}

\subsection{Elasticsearch}

\section{Search Engine}

\subsection{Term Frequency}

\subsection{Inverse Document Frequency}

\subsection{Document Normalization}

\subsection{Document Score}
\label{sec:doc-score}

\subsection{Multiple Term Query}

\section{Relevance Feedback}

\subsection{Explicit vs Implicit Feedback}

\subsection{Pseudo-Relevance Feedback}

\section{Query Expansion}
\label{sec:query-expansion}

[Kullback-Leibler Divergence]


[Kullback-Leibler Divergence Example]
