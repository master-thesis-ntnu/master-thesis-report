\chapter{Background}
\label{ch:background}

\section{Open Source Search Engine Solutions}
The increasing amount of data, makes data more difficult to utilize.
To handle the problem search engines spezialized in lagre amounts of data have been developed.
Elasticsearch\footnote{\url{https://www.elastic.co/products/elasticsearch}},
Solr\footnote{\url{http://lucene.apache.org/solr/}} and Xapian\footnote{\url{https://xapian.org/}} are a few widely used open source search engines.

Elasticsearch

Solr is based on the indexing and search technology called Lucene.

Xapian

None of the open source databases deliver personalized search capabilities.

\section{Technology}

\subsection{Lucene}

\subsection{Elasticsearch}

\section{Search Engine}

\subsection{Term Frequency}

\subsection{Inverse Document Frequency}

\subsection{Document Normalization}

\subsection{Document Score}
\label{sec:doc-score}

\subsection{Multiple Term Query}

\section{Relevance Feedback}

\subsection{Explicit vs Implicit Feedback}

\subsection{Pseudo-Relevance Feedback}

\section{Query Expansion}
\label{sec:query-expansion}

[Kullback-Leibler Divergence]


[Kullback-Leibler Divergence Example]
