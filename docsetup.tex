\usepackage{setspace}
\usepackage{graphicx}
\usepackage{amssymb}
\usepackage{mathrsfs}
\usepackage{amsthm}
\usepackage{amsmath}
\usepackage{color}
\usepackage[Lenny]{fncychap}
\usepackage[pdftex,bookmarks=true]{hyperref}
\usepackage[pdftex]{hyperref}
\usepackage{listings} % Allows code-listings
\usepackage{courier} % proper monospace font for code
\hypersetup{
    colorlinks,%
    citecolor=black,%
    filecolor=black,%
    linkcolor=black,%
    urlcolor=black
}
\usepackage[font=small,labelfont=bf]{caption}
\usepackage{fancyhdr}
\usepackage{times}
%\usepackage[intoc]{nomencl}
%\renewcommand{\nomname}{List of Abbreviations}
%\makenomenclature
\usepackage{natbib}
\usepackage{float}
%\floatstyle{boxed}
\restylefloat{figure}

% Algorithm highlighting
\usepackage{algorithm}
\usepackage{algpseudocode}

%\usepackage[number=none]{glossary}
%\makeglossary
%\newglossarytype[abr]{abbr}{abt}{abl}
%\newglossarytype[alg]{acronyms}{acr}{acn}
%\newcommand{\abbrname}{Abbreviations}
%\newcommand{\shortabbrname}{Abbreviations}
%\makeabbr
%\newcommand{\HRule}{\rule{\linewidth}{0.5mm}}

\renewcommand*\contentsname{Table of Contents}

\pagestyle{fancy}
\fancyhf{}
\renewcommand{\chaptermark}[1]{\markboth{\chaptername\ \thechapter.\ #1}{}}
\renewcommand{\sectionmark}[1]{\markright{\thesection\ #1}}
\renewcommand{\headrulewidth}{0.1ex}
\renewcommand{\footrulewidth}{0.1ex}
\fancypagestyle{plain}{\fancyhf{}\fancyfoot[LE,RO]{\thepage}\renewcommand{\headrulewidth}{0ex}}

% Add support for math equations (cequation)
\DeclareCaptionType{cequation}[][List of equations]
\captionsetup[cequation]{labelformat=empty}

\definecolor{gray75}{gray}{0.75}
\definecolor{gray1}{gray}{0.97}
\definecolor{gray2}{gray}{0.90}
\definecolor{gray3}{gray}{0.80}
\definecolor{gray4}{gray}{0.63}

\lstset{
    basicstyle=\footnotesize\ttfamily, % Standardschrift
    %numbers=left,               % Ort der Zeilennummern
    numberstyle=\tiny,          % Stil der Zeilennummern
    %stepnumber=2,               % Abstand zwischen den Zeilennummern
    numbersep=5pt,              % Abstand der Nummern zum Text
    tabsize=2,                  % Groesse von Tabs
    extendedchars=true,         %
    breaklines=true,            % Zeilen werden Umgebrochen
    keywordstyle=\color{red},
    frame=b,
 %  keywordstyle=[1]\textbf,    % Stil der Keywords
 %  keywordstyle=[2]\textbf,    %
 %  keywordstyle=[3]\textbf,    %
 %  keywordstyle=[4]\textbf,   \sqrt{\sqrt{}} %
 %  stringstyle=\color{white}\ttfamily, % Farbe der String
    showspaces=false,           % Leerzeichen anzeigen ?
    showtabs=false,             % Tabs anzeigen ?
    xleftmargin=17pt,
    framexleftmargin=17pt,
    framexrightmargin=5pt,
    framexbottommargin=4pt,
    %backgroundcolor=\color{lightgray},
    showstringspaces=false      % Leerzeichen in Strings anzeigen ?
}

\usepackage{caption}
\DeclareCaptionFont{white}{\color{white}}
\DeclareCaptionFormat{listing}{\colorbox[cmyk]{0.43, 0.35, 0.35,0.01}{\parbox{\textwidth}{\hspace{15pt}#1#2#3}}}
\captionsetup[lstlisting]{format=listing,labelfont=white,textfont=white, singlelinecheck=false, margin=0pt, font={bf,footnotesize}}

%%%%%%%%%%%%%%%%%%%% JSON code highlighting %%%%%%%%%%%%%%%%%%%%%%%%%%%%%
% Monospace font
%\usepackage{bera}
\usepackage{xcolor}

% Define colors for JSON highlighting
\colorlet{punct}{red!60!black}
\definecolor{background}{HTML}{EEEEEE}
\definecolor{delim}{RGB}{20,105,176}
\colorlet{numb}{magenta!60!black}

\lstdefinelanguage{json}{
    basicstyle=\normalfont\ttfamily,
    numbers=left,
    numberstyle=\scriptsize,
    stepnumber=1,
    numbersep=8pt,
    showstringspaces=false,
    breaklines=true,
    frame=lines,
    backgroundcolor=\color{background},
    literate=
     *{0}{{{\color{numb}0}}}{1}
      {1}{{{\color{numb}1}}}{1}
      {2}{{{\color{numb}2}}}{1}
      {3}{{{\color{numb}3}}}{1}
      {4}{{{\color{numb}4}}}{1}
      {5}{{{\color{numb}5}}}{1}
      {6}{{{\color{numb}6}}}{1}
      {7}{{{\color{numb}7}}}{1}
      {8}{{{\color{numb}8}}}{1}
      {9}{{{\color{numb}9}}}{1}
      {:}{{{\color{punct}{:}}}}{1}
      {,}{{{\color{punct}{,}}}}{1}
      {\{}{{{\color{delim}{\{}}}}{1}
      {\}}{{{\color{delim}{\}}}}}{1}
      {[}{{{\color{delim}{[}}}}{1}
      {]}{{{\color{delim}{]}}}}{1},
}
\definecolor{javared}{rgb}{0.6,0,0} % for strings
\definecolor{javagreen}{rgb}{0.25,0.5,0.35} % comments
\definecolor{javapurple}{rgb}{0.5,0,0.35} % keywords
\definecolor{javadocblue}{rgb}{0.25,0.35,0.75} % javadoc

\lstset{language=Java,
basicstyle=\ttfamily,
keywordstyle=\color{javapurple}\bfseries,
stringstyle=\color{javared},
commentstyle=\color{javagreen},
morecomment=[s][\color{javadocblue}]{/**}{*/},
numbers=left,
numberstyle=\tiny\color{black},
stepnumber=1,
numbersep=10pt,
tabsize=4,
showspaces=false,
showstringspaces=false}
