\chapter*{Sammendrag}
Dagens webtjenester genererer enorme datamengder.
Brukerene av disse tjenestene forventer at søkene leverer relevante søkeresultater umiddelbart.
En mye brukt teknikk i søkemotorer kalles term frequency-inverse document frequency(TF-IDF),
som er i stand til å levere relevante søk fort.
I dag inneholder enkelte tjenester så mye informasjon at det trengs mer avanserte metoder for å levere relevante søkeresultater.
En måte å forbedre søke på er ved å utvide søket til brukeren,
men en utfordring er å finne relevant informasjon som kan brukes i en søkeutvidelse.
Ved å kombinere søkeutvidelse med pseudo-relevant tilbakemelding har man funnet ut at søket vil bli mer relevant.
I tillegg til relevante søk er det viktig at søket går fort.
Forskning gjort av Google \cite{google-marissa} viste at et halvt sekund lenger ventetid på et søkeresultat
førte til markant mindre trafikk hos Google.

Denne oppgaven undersøker hvordan søkeutvidelse kan bli implementert sammen med pseudo-relevanse
for å levere mer relevante resultater.
Dette i seg selv er ikke ny forskning,
men mye av forskningen på feltet fokuserer ikke på hastighet.
Implementasjonen beskrevet i denne oppgaven vil ha et hovedfokus på hastighet,
og vil ha et krav om at søkeresultatene skal være tilgjengelig hos brukere innen 100 millisekunder.
100 millisekunder forsinkelse er det øverste taket før en bruker vil oppfatte forsinkelsen.
