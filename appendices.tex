\chapter{Appendix}

\section{Elasticsearch Query}
\begin{lstlisting}[language=json, caption={Initial query to retrive query expansion data}, label={ap:initial-query}]
{
    query : {
        terms : {
          tags : [...]
        }
    },
    size: 50,
    aggregations : {
        terms: {
            terms : {
                field : "tags.keyword",
                size: 50
            }
        },
        term_count: {
            value_count: {
                field: "tags.keyword"
            }
        }
    }
}
\end{lstlisting}

\begin{lstlisting}[language=json, caption={Elasticsearch term search}, label={ap:final-query}]
{
    query : {
        terms : {
          tags : [...]
        }
    }
}
\end{lstlisting}

\section{Flickr Data Representation in Elasticsearch}
\begin{lstlisting}[language={json}, caption={Internal photo data representation in elasticsearch}, label={ap:flickr-data}]
{
    id: id,
    title: "title",
    description: "description",
    tags: [
      "blue",
      "sky"
    ],
    dateuploaded: "1489805142",
    urls: [
      "url1",
      "url2"
    ]
}
\end{lstlisting}

\section{Elasticsearch Static Mapping}
\begin{lstlisting}[language={json}, caption={The static Elasticsearch mapping used on the photo index in the experiment setup.}, label={ap:elasticsearch-mapping}]
{
  "mappings": {
     "photo": {
        "properties": {
           "dateuploaded": {
              "type": "text",
              "fields": {
                 "keyword": {
                    "type": "keyword",
                    "ignore_above": 256
                 }
              }
           },
           "description": {
              "type": "text",
              "fields": {
                 "keyword": {
                    "type": "keyword",
                    "ignore_above": 256
                 }
              }
           },
           "id": {
              "type": "text",
              "fields": {
                 "keyword": {
                    "type": "keyword",
                    "ignore_above": 256
                 }
              }
           },
           "tags": {
              "type": "text",
              "fields": {
                 "keyword": {
                    "type": "keyword",
                    "ignore_above": 256
                 }
              }
           },
           "title": {
              "type": "text",
              "fields": {
                 "keyword": {
                    "type": "keyword",
                    "ignore_above": 256
                 }
              }
           },
           "urls": {
              "type": "text",
              "fields": {
                 "keyword": {
                    "type": "keyword",
                    "ignore_above": 256
                 }
              }
           }
        }
     }
  }
}
\end{lstlisting}

\section{Single Term Query}
[?? add code]
\begin{lstlisting}[language={java}, caption={Java code used to search for a single term.}, label={ap:single-term-query}]

\end{lstlisting}

\section{Multiple Term Query}
\begin{lstlisting}[language={java}, caption={Java code used to search for multiple terms.}, label={ap:multiple-term-query}]
SearchResponse searchResponse = client
    .prepareSearch(INDEX_NAME)
    .setQuery(QueryBuilders.termsQuery("tags", termsQuery))
    .setSize(searchResultSize)
    .get();

return searcchResponse;
\end{lstlisting}

\section{Query to Retrieve the Number of Occurences for a Given Term}
\begin{lstlisting}[language={java}, caption={Java code used retrieve the number of occurences for a given term in the collection.}, label={ap:number-of-occurences-query}]
SearchResponse searchResponse = client.prepareSearch(INDEX_NAME)
    .setQuery(QueryBuilders.termQuery("tags", term))
    .setSize(0)
    .get();

long numberOfTimesInCollection = searchResponse
    .getHits()
    .getTotalHits();

return numberOfTimesInCollection;
\end{lstlisting}

\section{Field Stats Query}
\begin{lstlisting}[language={java}, caption={Java code used retrieve the total number of terms in a field in the collection.}, label={ap:field-stasts-query}]
FieldStatsResponse fieldStatsResponse = client
    .prepareFieldStats()
    .setFields(FieldNames.TAGS_FIELD_NAME)
    .get();

long numberOfTermsInCollection = fieldStatsResponse
    .getAllFieldStats()
    .get(FieldNames.TAGS_FIELD_NAME)
    .getSumTotalTermFreq();

return numberOfTermsInCollection;
\end{lstlisting}

\section{ApacheBench}
\label{ap:apache-benchmark}

ApacheBench \cite{apache-benchmark} were used to measure response times.
The bullet list below contains a description of the command at the end of this section.

\begin{itemize}
  \item \texttt{-n} defines the number of times to send a request.
  \item \texttt{-c} defines the number of concurrent request.
  \item \texttt{-l} is used to allow responses of different sizes
  \item \texttt{localhost} have to be changed to the IP or domain name of the web server.
  \item The url contains two parameters, \texttt{q} and \texttt{size}.
  \texttt{q} is the acutal search query and \texttt{size} defines the size of the search result.
\end{itemize}

\texttt{ab -n 10000 -c 10 -l http://localhost/expansion?q=sky+blue\&size=10}
