\chapter*{Abstract}
Todays web services generate large amounts of data.
Users expect these services to have a search which returns relevant search results instantaneously.
Term frequency-inverse document frequency (TF-IDF) is a common technique used in search engines to deliver relevant search results fast.
However,
with the increasing amounts of data users expect the search results to deliver even more relevant search results.
Improving the search results can be done by expanding the user's query.
Providing relevant information for query expansion may be a challenge,
but using a technique called pseudo-relevance feedback have shown promissing results.
Relevant search results are important, but equally important is speed.
Research by Google \cite{google-marissa} found that 0.5 seconds increased load times resulted signifiant less traffic.

This thesis investigates how query expansion can be implemented together with pseudo-relevance,
to deliver more relevant search results.
Research on how to improve search results is not new,
but the focus is rarely speed.
The implementation will focus on speed and will have a requirement to deliver search results within 100 ms.
100 ms is the maximum acceptable amount of time before users will notice the delay.

% Use this for reference for google speed?
% http://dl.acm.org/citation.cfm?id=1496103
